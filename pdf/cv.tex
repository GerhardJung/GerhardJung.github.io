% !TEX root = ./cv_main.tex

\section*{Curriculum Vitae}
\label{sec:cv}


\subsection*{Personal Information}

\begin{tabular}{rl}
	\textsc{Date of birth:} & Wiesbaden, Germany  | April 1st, 1992 \\
	\textsc{Email:}     & \href{gerhard.jung.physics@gmail.com}{\small \url{gerhard.jung.physics@gmail.com}}  \\

\textsc{Homepage:}     & \href{https://gerhardjung.github.io/}{\small \url{https://gerhardjung.github.io/}}, \href{https://scholar.google.com/citations?user=0Wk52qAAAAAJ&hl=en}{\small \url{Google Scholar}}\\
	
\end{tabular}

\subsection*{Main Areas of Research}

I develop next-generation coarse-graining techniques for soft matter systems, drawing on methods from statistical physics and machine learning. My research focuses on complex dynamics in glassy materials and non-equilibrium active matter.

%Section: Work Experience at the top
\subsection*{Academic Experience}
\begin{tabular}{r|p{13cm}}
		\textsc{11/2025  }& \textsc{University of Innsbruck (Austria) } \\-&\emph{Senior Scientist}\\
	Present&\footnotesize{Project funded by Austrian Science Fund}\\
	&\footnotesize{Area of study: Modeling and exploiting non-equilibrium soft matter}\\		\multicolumn{2}{c}{}\\
	\textsc{12/2023  }& \textsc{LIPhy, Grenoble (France) } \\-&\emph{Postdoctoral researcher}\\
	10/2025&\footnotesize{Advisor: Eric Bertin, Misaki Ozawa}\\
	&\footnotesize{Area of study: Decentralized machine learning, statistical physics of social agents}\\		\multicolumn{2}{c}{}\\
		\textsc{10/2021  }& \textsc{CNRS, Montpellier (France) } \\-&\emph{Postdoctoral researcher}\\
	12/2023&\footnotesize{Advisor: Prof. Ludovic Berthier, Prof. Giulio Biroli}\\
	&\footnotesize{Area of study: Glass transition, machine learning, amorphous defects}\\
		\multicolumn{2}{c}{}\\
		\textsc{06/2021  }& \textsc{University of Kyoto (Japan) } \\-&\emph{JSPS Fellow}\\
	\textsc{10/2021  }&\footnotesize{Advisor: Prof. Ryoichi Yamamoto}\\
	&\footnotesize{Area of study: Active particles in viscoelastic media}\\
		\multicolumn{2}{c}{}\\
	\textsc{03/2019  }& \textsc{University of Innsbruck (Austria) } \\-&\emph{Postdoctoral researcher}\\
	\textsc{05/2021}&\footnotesize{Advisor: Prof. Thomas Franosch}\\
	&\footnotesize{Area of study: Glass transition, crystallization, confined geometry}\\
	\multicolumn{2}{c}{}\\
	
		\textsc{05/2018  }& \textsc{Durham University (UK)} \\\textsc{}-&\emph{Visiting researcher}\\10/2018&\footnotesize{  Advisor: Prof. Suzanne Fielding }\\
	&\footnotesize{Area of study: Soft glassy materials, yielding transition, rheology}\\
	\multicolumn{2}{c}{}\\
	
	\textsc{10/2014 }& \textsc{University of Mainz (Germany) } \\\textsc{}-&\emph{Doctoral and postdoctoral researcher}\\02/2019 &\footnotesize{Advisor: Prof. Friederike Schmid}
	\\&\footnotesize{Area of study: Non-Markovian dynamics, systematic coarse-graining, rheology, nonequilibrium
		dynamics}
	\\\multicolumn{2}{c}{}
	

	
\end{tabular}

%Section: Education
\subsection*{Education}
\begin{tabular}{r|p{14cm}}	
	
		\textsc{01/10/2014 }& \textsc{University of Mainz } \\\textsc{}-&\emph{Doctor rerum naturalium} (fast-track program )\\ 13/12/2018&\footnotesize{Advisor: Prof. Friederike Schmid}
	\\&\footnotesize{Thesis title: \emph{`Frequency-dependent phenomena and memory in soft matter systems.'}}
	\footnotesize{Grade: 0.7 (\emph{summa cum laude})}
	\\\multicolumn{2}{c}{}\\
	
		\textsc{21/12/2017}& \textsc{University of Mainz } \\\textsc{}&\emph{Master of Science}\\&\footnotesize{Advisor: Prof. Friederike Schmid}
	\\&\footnotesize{Thesis title: \emph{`Frequency-dependent hydrodynamic interaction between two
			solid spheres.'}}
	\footnotesize{Grade: 1.0}
	\\\multicolumn{2}{c}{}\\
	
	\textsc{24/10/2011}& \textsc{University of Mainz } \\ - &\emph{Bachelor of Science}\\15/08/2014&\footnotesize{Advisor: Prof. Friederike Schmid}
\\&\footnotesize{Thesis title: \emph{`Phase diagrams of model lipid bilayers.'}}
\footnotesize{Grade: 1.0 (\emph{with distinction})}
\\\multicolumn{2}{c}{}\\
	

	\textsc{12/08/2002 } &  \textsc{Gymnasium Eltville}\\
		- &  \emph{High School}\\
		07/06/2011 &
	
\end{tabular}


\subsection*{Awards}
\begin{tabular}{rp{14cm}}
	\textsc{2025:}& Principal Investigator of the Austria Science Fund (funding amount  450,000€)\\
	\textsc{2025:}& Editors' Suggestion of the publication `Kinetic Theory of Decentralized Learning for Smart Active Matter' in Physical Review Letters \\
	\textsc{2021:}& Highlighted as ‘Emerging Leader 2021’ by the Journal of Physics: Condensed Matter\\
	\textsc{2021:}& JSPS `short-term postdoctoral fellowship'\\
		\textsc{2019:}& \emph{Dr. rer. nat.} with `summa cum laude'\\
\textsc{2016 - 2019:}& {Member of the ‘Graduate School of Excellence Materials Science in Mainz’
 } \\

 \textsc{2013 - 2015:} & {Admission to the ‘Studienstiftung des deutschen Volkes’ (scholarship)} \\

\textsc{2013 - 2014:}& {Recipient of the ‘Deutschlandstipendium’ (scholarship)
}
\\\multicolumn{2}{c}{}\\
	
\end{tabular}

\subsection*{Teaching and Mentoring}
\begin{tabular}{r|p{13cm}}
	\textsc{Lecturer}& {Advanced statistical physics
	} 
	\\\multicolumn{2}{c}{}\\
	\textsc{Tutorials}& Mathematical methods of physics, modelling (computer science), statistical
	physics, electrodynamics, computer simulations in statistical physics
	\\\multicolumn{2}{c}{}\\
	
	\textsc{Supervisor}& {Co-supervision of one PhD student}\\
	& {Co-supervision of one Master student}\\
	& {Supervision of three Bachelor students and one Master student}
	\\\multicolumn{2}{c}{}\\	
\end{tabular}

\subsection*{Organization of Academic Events}

	\begin{tabular}{rl}
			2022: & {AISSAI} Workshop on `Machine Learning Glassy Dynamics'  (5 days)\\
			2018: & CECAM workshop on ‘Dynamic coarse-graining and memory effects in soft matter systems” (2 days)\\
	2017: & SFB TRR146 students retreat (5 days)
	 
\end{tabular}

%Section: Languages
\subsection*{Languages}
\begin{tabular}{rl}
	\textsc{English:}& fluent\\
	\textsc{German:}& mother tongue\\
	\textsc{French:}& advanced 
\end{tabular}


\subsubsection*{Review activities}
 Reviewer for international journals: Physical Review Letters, Nature Communications, Communications Physics, Journal of Chemical Physics, Europhysics Letters, Physical Review E, Journal of Statistical Physics, SciPost Physics,  Frontiers in Physics, Molecular Simulation, Macromolecular Theory and Simulations


